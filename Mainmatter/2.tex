% !TEX root = ../main.tex
\begin{boks}{Sætning 1.1 (Taylors Sætning med restled)}
  Lad $k \in \mathbb{N}$, $x_0, x \in A \subset \mathbb{R}$, $x \neq x_0$ og $f \ : \ A \rightarrow \mathbb{R}$ opfylde, at den $j$'te afledede $f^{(j)}$ af $f$ eksisterer og er kontinuert på det lukkede interval mellem $x_0$ og $x$ og differntiabel på det åbne interval mellem $x_0$ og $x$ for alle $j \leq k$.
  Så findes et punkt $c$ strengt mellem $x_0$ og $x$ så
  \begin{align}\label{(1)}
    f(x) = \sum_{j = 0}^k \frac{f^{(j)}(x_0)}{j!}(x - x_0)^j + \frac{f^{(k + 1)}(c)}{(k + 1)!}(x - x_0)^{k + 1}.
  \end{align}
\end{boks}
\begin{proof}
  For $x \neq x_0$ findes netop én løsning $M \in \mathbb{R}$ til ligningen
  \begin{align*}
    f(x) = \sum_{j = 0}^k\frac{f^{(j)}(x_0)}{j!}(x - x_0)^j + M(x - x_0)^{k + 1}.
  \end{align*}
  Vi vil vise, at $M = \frac{f^{(k + 1)}(c)}{(k + 1)!}$ for et $c$ mellem $x_0$ og $x$. For at vise dette, definierer vi funktionen $F \ : \ [a,b] \rightarrow \mathbb{R}$ på følgende måde:
  \begin{align*}
    F(t) = \sum_{j = 0}^k \frac{f^{(j)}(t)}{j!}(x - t)^j + M(x - t)^{k + 1}.
  \end{align*}
Ifølge antagelserne er $F$ kontinuert på det lukkede intercal mellem $x_0$ og $x$ og differentiable på det åbne intercal mellem $x_0$ og $x$.
Vi har valgt $M$, således at $F(x_0) = f(x)$, og vi har desuden, at $F(x) = \frac{f^{(0)}(x)}{0!}0^0 + \sum_{j = 1}^k \frac{f^{(j)}(x)}{j!} \cdot 0^j + M \cdot 0^{k + 1} = f(x)$. Dermed kan vi anvende Middelværdisætningen på $F$ på intervallet mellem $x_0$ og $x$ og får eksistensen af et $c$ mellem $x_0$ og $x$ så
\begin{align*}
  F'(c) = \frac{F(x) - F(x_0)}{x - x_0} = 0.
\end{align*}
Men
\begin{align*}
  F'(c) = f'(c) + \sum_{j = 1}^k \left( \frac{f^{(j + 1)}(c)}{j!}(x - c)^j -
  \frac{f^{(j)}(c)}{(j - 1)!}(x - c)^{j - 1} \right) -
  M(k + 1)(x - c)^k,
\end{align*}
hvor alle led umiddelbart går ud med hinanden parvist, borset fra de to led i
\begin{align*}
  \frac{f^{(k + 1)}(c)}{k!}(x - c)^k - M(k + 1)(x - c)^k,
\end{align*}
som dermed må smumme til $0$, da $F'(c) = 0$, og dermed er
\begin{align*}
  M = \frac{f^{(k + 1)}(c)}{(k + 1)k!} = \frac{f^{(k + 1)}(c)}{(k + 1)!}
\end{align*}
som påstået.
\end{proof}

\subsection{Taylors Sætning for funktioner af flere variable}
Før vi formulerer og beviser Taylors Sætning for funktioner af flere variable, vil vi først introducere en meget nyttig og - anvendt notation.
\begin{boks}{Definition 2.1 (Multi-indeks-notation)}
  Et \textit{multi-indeks} er en $n$-tupel af ikke-negative heltal $\alpha \in \mathbb{N}_0^n$. For to multi-indices $\alpha = (\alpha_1, \alpha_2, \ldots, \alpha_n)$, $\beta = (\beta_1, \beta_2, \ldots, \beta_n) \in \mathbb{N}_0^n$ og et $x = (x_1, x_2, \ldots, x_n) \in \mathbb{R}^n$ defineres:
  \begin{enumerate}[label = \arabic*.]
    \item Sum/differens: $\alpha \pm \beta = (\alpha_1 \pm \beta_1, \alpha_2 \pm \beta_2, \ldots, \alpha_n \pm \beta_n).$
    \item  Absolutværdi: $|\alpha| = \sum_{i = 1}^n \alpha_i.$
    \item Fakultet: $\alpha! = \prod_{i = 1}^n \alpha_i!.$
    \item Potens: $x^\alpha = \prod_{i = 1}^n x_i^{\alpha_i}.$
    \item Partielt afledet af højere orden: $\partial^\alpha = \prod_{i = 1}^n \partial_i^{\alpha_i}$ hvor $\partial_i^{\alpha_i} = \frac{\partial^{\alpha_i}}{\partial x_i^{\alpha_i}}.$
  \end{enumerate}
\end{boks}

\begin{boks}{Sætning 2.2 (Taylors Sætning med restled)}
  Lad $k \in \mathbb{N}$, $x,x_0 \in A \subset \mathbb{R}^n$, $x \neq x_0$ og $f \ : \ A \rightarrow \mathbb{R}$ opfylde, at $\partial^\alpha f$ eksisterer og er differentiabel på linjestykket $L = \{(1 - t)x_0 + tx \in \mathbb{R}^n \ | \ t \in [0,1]\}$ mellem $x_0$ og $x$ for $|\alpha| \leq k$.
  Så eksisterer et $y \in \{(1 - t)x_0 + tx \in \mathbb{R}^n \ | \ t \in (0,1)\}$ så
  \begin{align*}
    f(x) = \sum_{|\alpha| \leq k} \frac{\partial^\alpha f(x_0)}{\alpha!}(x - x_0)^\alpha + \sum_{|\alpha| = k + 1}\frac{\partial^\alpha f(y)}{\alpha!}(x - x_0)^\alpha.
  \end{align*}
\end{boks}
\begin{proof}
  Da $\partial^\alpha f$ er antaget differentiabel på $L$ for alle $|\alpha| \leq k$, så må $L \subset A$ og $x_0, x \in A$ være indre punkter, og vi kan finde et $r > 0$ så $F \ : \ (-r, 1 + r) \rightarrow \mathbb{R}$ givet ved $F(t) = f((1-t)x_0 + tx)$ er veldefineret og $k + 1$ gange differentiabel på $[0,1]$. Dermed kan vi anvende Sætning 1.1 på $F$ med $x_0 = 0$ og $x = 1$:
  \begin{align*}
    F(1) = \sum_{j = 0}^k \frac{F^{(j)}(0)}{j!} + \frac{F^{(k + 1)}(c)}{(k + 1)!},
  \end{align*}
  hvor $c \in (0,1)$. Kædereglen giver nu:
  \begin{align*}
    F^{(j)}(t) = \sum_{|\alpha| = j} \frac{j!}{\alpha!}(\partial^\alpha f)((1-t)x_0 + tx)(x - x_0)^\alpha,
  \end{align*}
  se Opgave 1, hvoraf resultatet følger med $y = (1 - c)x_0 + cx$.
\end{proof}

% \subsection{Lokale ekstrema for funktioner af flere variable}
% \begin{boks}{Definition 4.4 (Hesse-matrix)}
%   Lad $x \in A \subset \mathbb{R}^n$, $A$ åben og $f \in C^2(A)$. Så kaldes matricen $H(x)$ givet ved
%   \begin{align*}
%     h_{ij}(x) = \frac{\partial^2 f(x)}{\partial x_i \partial x_j}, \quad
%     H(x) = (h_{ij}(x))_{i,j = 1}^n
%   \end{align*}
%   for \textit{Hesse-matricen} for $f$ i $x$.
% \end{boks}
%
% \begin{boks}{Lemma 4.5}
%   Lad $x,x_0 \in A \subset \mathbb{R}^n$ opfylde at $A$ er åben og at $L = \{(1-t)x_0 + tx \in \mathbb{R}^n \ | \ t \in (0,1)\}$ så
%   \begin{align*}
%     f(x) = f(x_0) + \langle \nabla f(x_0), x - x_0 \rangle + \frac{1}{2} \langle x - x_0, H(y)(x - x_0) \rangle.
%   \end{align*}
% \end{boks}
% \begin{proof}
%   Beviset er blot Taylors sætning med restled for $k = 1$:
%   \begin{align}
%     f(x) &= \sum_{| \alpha | \leq 1} \frac{\partial^\alpha f(x_0)}{\alpha!} (x - x_0)^\alpha +
%     \sum_{| \alpha | = 3} \frac{\partial^\alpha f(y)}{\alpha!} (x - x_0)^\alpha \nonumber\\
%     &= \frac{\partial^0 f(x_0)}{0!}(x - x_0)^0 +
%     \frac{\partial^1 f(x_0)}{1!}(x - x_0)^1 +
%     \frac{1}{2}\frac{\partial^2 f(y)}{\partial x_1^2}(x - x_0)^2 +
%     \frac{1}{2}\frac{\partial^2 f(y)}{\partial x_2^2}(x - x_0)^2\nonumber\\
%     &\qquad+
%     \frac{1}{2}\frac{\partial^2f(y)}{\partial x_1 \partial x_2}(x - x_0) +
%     \frac{1}{2}\frac{\partial^2 f(y)}{\partial x_1 \partial x_2}(x - x_0) +
%     \frac{1}{2}\frac{\partial^2 f(y)}{\partial x_2 \partial x_1}(x - x_0)\nonumber\\
%     &= f(x_0) + f'(x_0)(x - x_0) +
%     1/2 \left( \frac{\partial^2f(y)}{\partial x_1 \partial x_2}(x - x_0)^2 +
%     \frac{\partial^2 f(y)}{\partial x_2^2}(x - x_0)^2 + 2\frac{\partial^2 f(y)}{\partial x_1 \partial x_2} \right)\label{taylor_omskrivning}
%   \end{align}
%   Her kommer nogle mellem trin, de følgende matricer bruges til forståelse
%   \begin{align*}
%     H(y) =
%     \begin{bmatrix}
%       \frac{\partial^2 f(y)}{\partial x_1^2} &
%       \frac{\partial^2 f(y)}{\partial x_2 \partial x_1} \\
%       \frac{\partial^2 f(y)}{\partial x_1 \partial x_2} &
%       \frac{\partial^2 f(y)}{\partial x_2^2}
%     \end{bmatrix}, \qquad
%     H(y)(x - x_0) =
%     \begin{bmatrix}
%       \frac{\partial^2 f(y)}{\partial x_1^2}(x - x_0) &
%       \frac{\partial^2 f(y)}{\partial x_2 \partial x_1}(x - x_0) \\
%       \frac{\partial^2 f(y)}{\partial x_1 \partial x_2}(x - x_0) &
%       \frac{\partial^2 f(y)}{\partial x_2^2}(x - x_0)
%     \end{bmatrix}.
%   \end{align*}
%   Ved brug af det ovenstående ser vi at
%   \begin{align*}
%     &\frac{1}{2} \langle x - x_0, H(y)(x - x_0) \rangle =\\
%     &\qquad \frac{1}{2}
%     \bigg( \frac{\partial^2 f(y)}{\partial x_1^2}(x- x_0)^2 +
%     \frac{\partial^2 f(y)}{\partial x_2 \partial x_1}(x- x_0)^2 +
%     \frac{\partial^2 f(y)}{\partial x_1 \partial x_2}(x- x_0)^2 +
%     \frac{\partial^2 f(y)}{\partial x_2^2}(x- x_0)^2 \bigg)
%   \end{align*}
%   Nu kan vi omskrive \eqref{taylor_omskrivning} til det følgende, ved brug af det ovenstående
%   \begin{align*}
%     f(x_0) + \langle \nabla f(x_0), x - x_0 \rangle + \frac{1}{2}\langle x - x_0, H(y)(x - x_0)\rangle
%   \end{align*}
%   hvilket fuldender beviset.
% \end{proof}
%
% \begin{boks}{Definition 4.6}
%   Antag at $A \subset \mathbb{R}^n$ er åben, at $f \in C^2(A)$, at $x_0 \in A$ er et kritisk punkt for $f$ og at alle Hesse-matricen $H(x_0)$'s egenværdier er positive.
%   Så er $x_0$ et lokalt minimumspunkt for $f$. Hvis alle $H(x_0)$'s egenværdier omvendt er negative, er $x_0$ et lokalt maksimumspunkt for $f$.
% \end{boks}
% \begin{proof}
%   Antag uden tab af generalitet at alle egeværdier for $H(x_0)$ er positive.
%   Da $H(x_0)$ er reel og symmestrisk er den selvadjungeret og spektralsætningen giver, at $H(x_0)$ har $n$ reelle, ortonormale egenvektorer $e_i$, $i = 1, \ldots, n$ med tilhørende egenværdier $\lambda_i$. Vi kan derfor for $z \in \mathbb{R}^n$ skrive
%   \begin{align*}
%     z = \sum_{i = 1}^n \langle z, e_i \rangle e_i
%   \end{align*}
%   og dermed
%   \begin{align*}
%     H(x_0)z = \sum_{i = 1}^n \langle z, e_i \rangle \lambda_i e_i
%   \end{align*}
%   eller
%   \begin{align*}
%     \langle z, H(x_0)z \rangle = \sum_{i = 1}^n \lambda_i \langle z, e_i \rangle^2.
%   \end{align*}
%   Lad $m = \min \{ \lambda_i \ | \ i \in \{ 1, \ldots, n \}\} > 0$. Så er
%   \begin{align*}
%     \langle z, H(x_0)z \rangle = \sum_{i = 1}^n \lambda_i \langle z, e_i \rangle^2 \geq \sum_{i = 1}^n m \langle z, e_i \rangle^2 = m \| z \|^2.
%   \end{align*}
%   Sæt $A(x) = H(y) - H(x_0)$, som afhænger af $x$ gennem $y \in \{ (1 - t)x_0 + tx \in \mathbb{R}^n \ | \ t \in (0,1) \}$.
%   Da $f \in C^2(A)$ er $\| A(x) \|_{HS}$ kontinuert og vi kan finde $r > 0$ så $x \in B_r(x_0) \backslash \{ x_0 \}$ medfører, at $\| A(x) \|_{HS} \leq \frac{1}{2}m$. Anvender vi nu Lemma 4.5 på $x \in B_r(x_0)$ fås
%   \begin{align*}
%     f(x) &= f(x_0) + \langle \nabla f(x_0), x - x_0 \rangle + \frac{1}{2}\langle x - x_0, H(y)(x - x_0)\rangle \\
%     &= f(x_0) + 0 + \frac{1}{2}\Big( \langle x - x_0, H(x_0)(x - x_0) \rangle + \langle x - x_0, A(x)(x - x_0) \rangle \Big)\\
%     &\geq f(x_0) + \frac{1}{2}\| x - x_0 \|^2 (m - \frac{1}{2}m) =
%     f(x_0) + \frac{m\| x - x_0 \|^2}{4},
%   \end{align*}
%   hvor vi igen brugte Cauchy-Schwartz.
% \end{proof}
