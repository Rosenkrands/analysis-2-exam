% !TEX root = ../main.tex
\begin{boks}{Definition 5.1}
  \begin{itemize}[label = $\ast$]
    \item $(X, d)$ metrisk rum
    \item $F \ : \ X \rightarrow X$ kaldes en kontraktion hvis $\exists \ \alpha \in [0,1)$ så
    $$d(F(x), F(y)) \leq \alpha d(x,y), \quad \forall x,y \in X$$
    \item Et punkt kaldes et fikspunkt for $F$ hvis $F(x) = x$
  \end{itemize}
\end{boks}
\begin{boks}{Sætning 5.2}
  $(X,d)$ fuldstændigt metrisk rum, $F \ : \ X \rightarrow X$ kontraktion så har $F$ et unikt fiksprunkt.
\end{boks}
\begin{proof}
  \textbf{Først viser vi unikhed}

  Antag $a,b \in X$ så $F(a) = a$ og $F(b) = b$.
  $$d(a,b) = d(F(a), F(b)) \leq \alpha d(a,b)$$
  $$d(a,b) \leq \alpha d(a,b)$$
  $$1 \leq \alpha$$
  hvilket er en modstrid. Derfor kan det konkluderes at hvis der findes et fikspunkt, så er det unikt.

  \textbf{Nu konstruerer vi et fikspunkt}

  Betragt følgen $ \{ y_n \}_{n \geq 1} \subset X$, hvor $y_1$ er vilkårlig og
  $$y_n \ := \ F(y_{n - 1}), \quad n \geq 2$$
  Betragt tilfældet hvor $y_2 = F(y_1) = y_1$, det vil betyde $y_1$ er vores fikspunkt og vi er færdige.

  Vi kan derfor antage at $d(y_2, y_1) > 0$.

  Vi viser to ting
  \begin{enumerate}[label = (\arabic*)]
    \item Følgen er Cauchy i $X$, derfor konvergerer den til $y \in X$
    \item $y$ er fikspunkt for $F$
  \end{enumerate}
  Vi starter med (1)

  For ethvert $\varepsilon > 0$ vil vi finde $N(\varepsilon) > 0$ så $ p \geq q \geq N(\varepsilon)$ medfører $d(y_p, y_q) < \varepsilon$.
  Med andre ord
  \begin{align}\label{kors_banach}
    d(y_q, y_{q + k}) < \varepsilon, \quad \forall k \geq 0, \ \forall q \geq N(\varepsilon)
  \end{align}
  Hvis $k \geq 0$ følger af trekantsuligheden at
  \begin{align}
    d(y_q, y_{q + k}) &\leq d(y_q,, y_{q + 1}) + d(y_{q + 1}, y_{q + k})\nonumber\\
    &\leq \sum_{i = 0}^{k - 1} d(y_{q + i}, y_{q + i + 1})\label{dobbelt_kors_banach}
  \end{align}

  For ethvert $n \leq 1$ har vi
  \begin{align*}
    d(y_n, y_{n + 1}) = d(F(y_{n - 1}), F(y_n)) \leq \alpha d(y_{n - 1}, y_n)
    \leq \ldots \leq \alpha^{n - 1} d(y_1, y_2), \quad \forall n \geq 1
  \end{align*}
  Det vil sige
  \begin{align*}
    d(y_{q + i}, y_{q + i - 1}) \leq \alpha^{q + i - 1} d(y_1, y_2), \quad \forall q \geq 1, \ i \geq 0
  \end{align*}
  Det ovenstående sammen med \eqref{dobbelt_kors_banach} medfører at
  \begin{align*}
    d(y_q, y_{q + k}) \leq \alpha^{q - 1} d(y_1, y_2) (1 + \ldots + \alpha^{k - 1}) \leq \frac{\alpha^{q - 1}}{1 - \alpha} d(y_1, y_2)
  \end{align*}
  Da $\alpha < 1$ følger det at $\lim \alpha^q = 0$.

  Vi vil gerne opfylde \eqref{kors_banach}
  \begin{align*}
    \frac{\alpha^{q - 1}}{1 - \alpha}d(y_1, y_2) &< \varepsilon \\
    \alpha^{q} &< \varepsilon \frac{\alpha(1 - \alpha)}{d(y_1, y_2)}
  \end{align*}
  vælg $N(\varepsilon)$ så det ovenstående er opfyldt.

  Det vil sige at $$ \lim_{n \rightarrow \infty} d(y_n, y_{infty}) = 0 $$

  \textbf{Vi viser nu (2)}

  For $n \geq 1$:
  \begin{align*}
    d(F(y), y) \leq d(F(y), F(y_n)) + d(F(y_n), y)
  \end{align*}
  men
  \begin{align*}
    d(F(y), F(y_n)) \leq \alpha d(y, y_n) \rightarrow 0
  \end{align*}
  og
  \begin{align*}
    d(F(y_n), y) = d(y_{n + 1}, y) \rightarrow 0
  \end{align*}
  afstanden mellem to punkter er nul hvis og kun hvis de to punkter er ens! Dermed konkluderer vi at
  $$ F(y) = y$$
  hvilket afslutter beviset.
\end{proof}
